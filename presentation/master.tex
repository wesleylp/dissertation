%******************************************************
% Header
%******************************************************

%Document type
\documentclass{beamer}

%Theme
%\usetheme{Madrid}

%Packages
\usepackage{etex}
\usepackage{mypresentation}							%SMT presentation style
\usepackage[T1]{fontenc}
\usepackage[utf8]{inputenc}
% \usepackage[utf8]{inputenc}							%Language/input coding
% \usepackage[brazil, english]{babel}					%Brazilian Portuguese Language Package
% \usepackage[T1]{fontenc}							%Hyphenation
% \usepackage{graphicx}								%Graphic package
% %\usepackage{subfig}									%Subfloats
% \usepackage{color}									%Color package
% \usepackage{lmodern}								%Solve font problems
\usepackage{amssymb}								%Math symbols
\usepackage{amsthm}									%Theorem
% \usepackage{multimedia}								%Multimedia package
% \usepackage{array}									%For tabular
%\usepackage{pgfplotstable}								%For extracting tables from csv
\usepackage{tikz}										%Tikz graphic library
\usetikzlibrary{patterns, calc}							%Tikz patterns and computations

\usepackage{multirow}
\usepackage{multicol}
\usepackage{booktabs}
\usepackage{subfig}
\usepackage{myMacros}
\usepackage{hyperref}

\graphicspath{{/home/wesley/Dropbox/Mestrado/dissertation/text/v0/images/}}

%******************************************************
% Math operators
%******************************************************
% \DeclareMathOperator{\tr}{tr}								%Trace
% \DeclareMathOperator{\diag}{diag}							%Diag
% \DeclareMathOperator*{\argmax}{arg\,max}					%Argmax
% \DeclareMathOperator*{\argmin}{arg\,min}					%Argmin
% \DeclareMathOperator{\sgn}{sgn}								%Signal function
% \newcommand{\Trm}{\mathrm{T}}								%Transpose
% \newcommand{\drm}{\mathrm{d}}								%Differential
% \newcommand{\erm}{\mathrm{e}}								%Exponential
% \newcommand{\jrm}{\mathrm{j}}								%Complex number

\newcommand\blfootnote[1]{%
	\begingroup
	\renewcommand\thefootnote{}\footnote{#1}%
	\addtocounter{footnote}{-1}%
	\endgroup
}


%******************************************************
% Title page
%******************************************************

%Title and subtitle
\title[Wesley L. Passos]{{\huge {Automatic {\it Aedes aegypti} Breeding Grounds Detection Using Computer Vision Techniques} }
	}

%Author
\author[]{
	Wesley Lobato Passos\\
	\vspace{5mm}
	\footnotesize{Eduardo A. B. da Silva \&
	Gabriel M. Araujo}
}



%Institute
\institute[SMT/COPPE/UFRJ]
{
	Signals, Multimedia, and Telecommunications Laboratory \newline
	COPPE/UFRJ
}

\date{February 27, 2019}
%******************************************************
% Section open
%******************************************************

\AtBeginSection[]
{
	\begin{frame}
		\frametitle{Sumário}
		%\tableofcontents[currentsection, hideallsubsections]
		\tableofcontents[currentsection]
	\end{frame}
}

%******************************************************
% Main Body
%******************************************************

%Document Beginning
\begin{document}

  %% Front page logos.
  \MyLogos{0.60cm}

  %Title frame
  \begin{frame}
    \titlepage
  \end{frame}

  %% Smaller logos for the other slides.
  \MyLogos{0.30cm}

  %Table of Contents
  \begin{frame}
    \frametitle{Outline}
    \tableofcontents[hideallsubsections]
  \end{frame}



	%%============================================================================
	%% SECTION - Introduction
	%%============================================================================
	\section{Introduction}

		\begin{frame}\frametitle{Motivation}

			\begin{itemize}
				\item The arboviruses ({\bf dengue}, {\bf chikungunya}, {\bf zika}, and {\bf yellow fever})
				are one of the leading global health problem;
				\item the {\it Aedes aegypti} is the main {\bf vector} of such diseases;
				\item the best form of combat is {\bf eliminating} possible mosquito {\bf foci} proliferation;
				\item it can be {\bf expensive} and {\bf time-consuming} (therefore, {\bf inefficient});
				\item to use {\bf aerial images} and {\bf videos} captured by an {\bf UAV};
				\item {\bf automatic detection} of regions and objects with {potential mosquito breeding sites} in order to assist health agents;
				\item to apply {\bf computer vision} and {\bf machine learning} techniques.
		\end{itemize}

	\end{frame}


	%%==============================================================================
	%% SECTION - The Aedes aegypti
	%%==============================================================================
	\section{The Aedes aegypti}


%%==============================================================================
%% SECTION - Vision Meets Unmanned Vehicles
%%==============================================================================
\section{Vision Meets Unmanned Vehicles}

	\subsection{Objects of interest}

		\begin{frame}\frametitle{Objects}
			\begin{figure}[htb]
				\centering
				\includegraphics[width=.1\linewidth]{garrafa1.png}
				\includegraphics[width=.2\linewidth]{garrafa2.png}
				\includegraphics[width=.1\linewidth]{pneu1.png}
				\includegraphics[width=.1\linewidth]{pneu3.png}\\
				\includegraphics[width=.1\linewidth]{pneu5.png}
				\includegraphics[width=.1\linewidth]{water1.png}
				\includegraphics[width=.1\linewidth]{water2.png}
				\includegraphics[width=.1\linewidth]{water3.png}
			% \includegraphics[width=.26\linewidth]{base4.png}
				\label{fig:objetos1}
			\end{figure}

			\begin{table}[]
				\resizebox{\textwidth}{!}{%
				\begin{tabular}{@{}cc@{}}
					\toprule
					\textbf{Code} & \textbf{Description}                                                       \\ \midrule
					A1            & Water tank connected to the grid (high tanks)                              \\
					A2            & Deposits at ground level (barrel, tub, drum, tank, well)                   \\
					B             & Mobile containers (vases/jars, plates, drippings, drinking fountains, etc) \\
					C             & Fixed deposits (tanks, gutters, slabs, etc.)                               \\
					D1            & Tires and other rolling materials                                          \\
					D2            & Garbage (plastic containers, bottles, cans, scraps)                        \\
					E             & Natural deposits (bromeliads, bark, tree holes)                            \\ \bottomrule
					\end{tabular}
					}
			\end{table}
		\end{frame}


	\subsection{Scenarios}

		\begin{frame}\frametitle{Scenarios}
			\begin{figure}[htb!]
				\centering
				\includegraphics[width=.35\linewidth]{base1.png}
				\includegraphics[width=.36\linewidth]{base2.png}\\
				\includegraphics[width=.35\linewidth]{base3.png}
			% \includegraphics[width=.26\linewidth]{../figures/base4.png}
				\label{fig:scenarios1}
			\end{figure}

		\end{frame}


\subsection{Camera Calibration: Zhang's Method}

% -----------------------------------------------------------------------------
% \frame{
%   \frametitle{Projection}
%   %
%   Considera-se que uma câmera mapeia um ponto $\Mbf' = [X,Y,Z,1]^\Trm$ do espaço no ponto $\mbf' = [u,v,1]^\Trm$ da imagem através de uma transformação projetiva da forma:
%   %
%   \begin{equation}
%   \label{eq:projection}
%   s\mbf' = \Asf[\Rsf~|~\tbf]\Mbf',
%   \end{equation}
%   %
%   onde $s$ é um fator de escala arbitrário, $\Rsf$ e $\tbf$ são, respectivamente, a matriz de rotação e o vetor de translação (parâmetros extrínsecos),que relacionam o sistema de coordenadas do mundo real com o sistema de coordenadas da câmera.
%   %
%   %
% }

% -----------------------------------------------------------------------------
\frame{
%TODO: make my own image.
  \frametitle{Pinhole Camera Model}
  %
  \begin{figure}[htb!]
    \center
    \includegraphics[width=.7\textwidth]{pinhole_camera_model.png}
    \label{fig:keypts_1} %\\[-0.2cm]
    \caption{Pinhole camera model. (source: OpenCV).}
%     \url{https://docs.opencv.org/3.4.1/d9/d0c/group__calib3d.html#ga3207604e4b1a1758aa66acb6ed5aa65d}
    \vspace{-3.5mm}
%     \href{https://docs.opencv.org/3.4.1/d9/d0c/group__calib3d.html#ga3207604e4b1a1758aa66acb6ed5aa65d}{[ref.: OpenCV documentation.]}
    \label{fig:undistort}
  \end{figure}
  %
  %
}

% -----------------------------------------------------------------------------
% \frame{
%   \frametitle{Parâmetros Intrínsecos}
%     $\Asf$ é a matriz de calibração da câmera (parâmetros intrínsecos), definida por:
%     \begin{equation}
%       \Asf =
%       \begin{bmatrix}
% 	\alpha & \gamma & u_0\\
% 	0 & \beta  & v_0\\
% 	0 &     0  & 1\\
%       \end{bmatrix},
%     \end{equation}
%     com $[u_0,v_0]^\Trm$ denotando as coordenadas do ponto principal,
%     $\alpha$ e $\beta$ os fatores de escala nos eixos $u$ e $v$ da imagem, respectivamente,
%     e $\gamma$ é a obliquidade (grau de cisalhamento) dos dois eixos da imagem.
% }

% -----------------------------------------------------------------------------
% \frame{
%   \frametitle{Compensação de Distorção Radial}
%    Câmeras convencionais geralmente possuem significantes distorções de lentes, especialmente distorção radial.
%
%    Sejam $(u,v)$ as coordenadas ideais (livre de distorção) do pixel na imagem e $(\breve{u},\breve{v})$ as coordenadas correspondentes reais de imagem observada.
%
%     Os pontos ideais são as projeções dos pontos do padrão de calibração de acordo com o modelo dado pela Eq.~\eqref{eq:projection}.
%
%    Analogamente, $(x,y)$ e $(\breve{x},\breve{y})$ são as coordenadas normalizadas nas imagens ideal (livre de distorção) e real (distorcida), respectivamente.
%
%    A distorção radial pode ser modelada como:
%    \begin{align}
%     \breve{x} &= x + x(k_1 r^2 + k_2 r^4) \\% + k_3 r^6)\\% + 2p_1 x y + p_2(r^2 + 2x^2), \\
%     \breve{y} &= y + y(k_1 r^2 + k_2 r^4), % + k_3 r^6),% + p_1(r^2 + 2y^2) + 2 p_2 x y,
%    \end{align}
% onde $k_1$ e $k_2$ são os coeficientes de distorção radial e $r^2 = (x^2 + y^2)$.
%
%
% }

% -----------------------------------------------------------------------------
% \frame{
%   \frametitle{Compensação de Distorção Radial}
%     %
%     O centro da distorção radial está localizado no ponto principal.
%
%     De $\breve{u} = \alpha \breve{x} + \gamma \breve{y} + u_0 $ e $\breve{v} = \beta \breve{y} + v_0$, assumindo $\gamma = 0$,
%     pode-se escrever que
%     %
%     \begin{align}
%       \label{eq:u_dist}
%       \breve{u} &= u + (u-u_0)(k_1 r^2 + k_2 r^4 ) \\
%       \label{eq:v_dist}
%       \breve{v} &= v + (v-v_0)(k_1 r^2 + k_2 r^4 ).
%     \end{align}
%
%
% }

% -----------------------------------------------------------------------------
% \frame{
%    \frametitle{Método de Zhang}
%     %
%     O método de Zhang consiste em fazer, primeiramente, uma estimativa grosseira dos parâmetros intrínsecos e extrínsecos da câmera para depois refiná-la através da estimativa da máxima verossimilhança.
%
%     Dadas $n$ imagens do padrão de calibração e considerando $m$ pontos neste padrão,
%     assumindo que as imagens dos pontos estão corrompidas por um ruído independente e identicamente distribuído, a estimativa de máxima verossimilhança pode ser obtida minimizando o funcional
%     %
%     \begin{equation}
%       \sum_{i=1}^{n}\sum_{j=1}^{m} \Vert \xbf_{ij} - \breve{\xbf}(\Asf, k_1, k_2, \Rsf_i, \tbf_i, \Xbf_j) \Vert^2
%       \label{eq:functional},
%     \end{equation}
%     %
%     onde $\breve{\xbf}(\Asf, k_1, k_2, \Rsf_i, \tbf_i, \Xbf_j)$ é a projeção do ponto $\Xbf_j$, de acordo com a Eq.~\eqref{eq:projection},
%     seguido da distorção modelada de acordo com as Eqs.~\eqref{eq:u_dist} e~\eqref{eq:v_dist}.
%
%     Minimizar o funcional em~\eqref{eq:functional} é um problema de otimização não-linear que pode ser resolvido através do algoritmo de Levenberg-Marquardt.
%
% }

% -----------------------------------------------------------------------------
\frame{
  \frametitle{Comparison: distorted frame $\times$ undistorted}

   %
    \begin{equation}
      \sum_{i=1}^{n}\sum_{j=1}^{m} \Vert \xbf_{ij} - \breve{\xbf}(\Asf, k_1, k_2, \Rsf_i, \tbf_i, \Xbf_j) \Vert^2
      \label{eq:functional},
    \end{equation}
    %

  \begin{figure}[htb!]
    \center
    \subfloat[]{\includegraphics[width=0.5\textwidth]{frame_420_pts.png}
      \label{fig:keypts_1}} %\\[-0.2cm]
      %(a)
      \subfloat[]{\includegraphics[width=0.5\textwidth]{frame_420_undistorted.png}%
      \label{fig:undistort_1}} %\\[-0.2cm]
      %(b)
%     \caption{Exemplo de correção de distorção radial pelo algoritmo de Zhang:
%     (a) imagem original distorcida com identificação das quinas detectadas no padrão de calibração;
%     (b) imagem retificada.}
    \label{fig:undistort}
  \end{figure}

}

\subsection{Annotation}

\begin{frame}\frametitle{Annotation}

\begin{figure}[htb]
	\centering
	\includegraphics[width=.98\columnwidth]{zframer_marking.png}
	\label{fig:zframer1}
\end{figure}
\end{frame}

%%==============================================================================
%% SECTION - Object Detection with Deep Learning
%%==============================================================================
	\section{Object Detection with Deep Learning}

		\subsection{Deep Learning}

			\begin{frame}\frametitle{Convolutional Neural Networks}
				\begin{figure}[htb]
					\centering
					\includegraphics[width=.4\columnwidth]{conv_eng.pdf}
					\includegraphics[width=6.5cm]{drone-persp2.png}
					\label{fig:zframer1}
				\end{figure}
			\end{frame}

%%==============================================================================
%% SECTION - Results and Discussions
%%==============================================================================
	\section{Results and Discussions}

		\subsection{Evaluation}

			\begin{frame}\frametitle{Intersection over Union (IoU)}
				\begin{itemize}
					\item Evaluate the model regarding object localization;
					\item Detections are considered as TP if the IoU $\geq$ threshold and as FP otherwise. hsabhs
				\end{itemize}
				%
				\begin{figure}[htb]
					\centering
					\includegraphics[width=.6\linewidth]{IoU.pdf}
					\label{fig:IoU}
				\end{figure}

			\end{frame}

%%==============================================================================
%% SECTION - Conclusion and Future Works
%%==============================================================================
	\section{Conclusion and Future Works}

		\begin{frame}
			\frametitle{Conclusion}
			\begin{itemize}
				\item To deliver a decision support system
				\item To indicate your geographic positions
				\item To facilitate scan abandoned or blocked areas
			\end{itemize}
		\end{frame}

		\begin{frame}
			\frametitle{THANK YOU!}
			\centering
			Wesley Lobato Passos\\
			wesley.lpassos@gmail.com
		\end{frame}



\end{document}