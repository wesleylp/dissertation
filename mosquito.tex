\chapter{The \textbf{Aedes aegypti}}
\label{chap:mosquito}
%
In order to fight against \textit{Aedes aegypti}, we must first know it better.
In this chapter, we address many aspects of the mosquito, including its
biology, diseases transmitted, and potential ground sites.
Also, we point some indicators and discuss plans and actions to combat the \textit{Aedes aegypti}.


\section{The vector biology}\label{sec:bio}
%
The \Aedes is considered one of the most dangerous mosquito species according to public health analysis, considering that it is the major vector of arbovirus transmission~\cite{ruckert2017}.
This mosquito species is well adapted to urban environment and inhabit mostly in domestic and peridomestic environments~\cite{Jansen2010}.
The high capacity of this vector to transmit diseases to humans is mostly due to a set of biological, ecological, and behavioral characteristics that promote a more direct contact with humans.
One of these characteristics is synanthropic behavior (live near human dwellings)~\cite{Jansen2010}.

The dispersion of \Aedes around the world began around the 16th century with the Portuguese maritime routes between Africa and the other continents.
Since then, always due to human transport, the mosquito has invaded many of the tropical and subtropical regions of the planet, establishing itself in the Americas, Southeast Asia, Southwest of the United States, islands of the Indian Ocean, and the north of Australia.
In areas outside the latitudes that comprise these regions, there have been some sporadic occurrences, even though the species shows an apparent difficulty in establishing a viable population in these places. However, considering the anticipated global climate change, the \Aedes may be able to expand its presence beyond customary regions~\cite{liu2016climate}.
The proliferation of the vector across the globe is related to the circulation of goods and people between various countries and continents.
The eggs of this species are particularly resistant and can survive long journeys and inhospitable environments, besides presenting great adaptability to artificial breeding sites~\cite{liu2016climate}.

The \Aedes is less than 1~cm long and black with white stripes,
as depicted in Figure~\ref{fig:aedes}.
%
%
\begin{figure}[b]
	\centering
	\includegraphics[width=.63\linewidth]{aedes.jpg}
	\caption[The {\it Aedes aegypti}]{The {\it Aedes aegypti}. Source:~\cite{web:aedesfiocruz}}
	\label{fig:aedes}
\end{figure}
%
The adult specimen lives on average 45 days, usually stings in the first hours of the morning and the late afternoon.
Studies by the Oswaldo Cruz Foundation (FIOCRUZ)
\abbrev{\id{FIOCRUZ}FIOCRUZ}{Oswaldo Cruz Foundation}
have shown that the female flies up to 1~km away from its eggs.
Moreover, the \Aedes feeds on plant sap.
The females are hematophagous, that is, they feed on blood as well.
As a consequence, when ingesting the blood of an infected host, it contracts the microorganism responsible for the diseases~\cite{lambrechts2012vector}.

The life cycle of {\it Aedes aegypti}, like other mosquito species, comprises four phases: egg, larva, pupa, and adult.
The first three phases develop exclusively in the aquatic environment, while the last phase occurs in the terrestrial/aerial environment.
Therefore, the existence of water and breeding suitable for their retention are essential for the mosquito development~\cite{Jansen2010}.
\Aedes breeding grounds are mostly small containers, either artificial or natural, in or near dwelling sites, which allow the storage of water (water tanks, buckets, ornamental fountains, plant dishes, water canisters for animals, tires, etc).
The preference for breeding sites near or in domestic environments relates to a set of habits that promote close contact with humans: synanthropic habits; endophilic (resting inside housing/animal facilities); and anthropophilic~\cite{Jansen2010}.
Hence, human activity is a determining factor for the reproduction and dissemination of the \textit{Aedes aegypti}.

Furthermore, the mosquito deposits the eggs under appropriate -- hot and water-filled -- places.
Under these conditions, the embryos take from two to three days to develop and hatch.
These embryos may weaken or die if, during this period, the eggs dry out, but if in the initial stage a perfect development is ensured, the eggs of the mosquito become resistant to drying and thus survive for periods ranging from some months to a year~\cite{lambrechts2012vector}.
This resistance is one of the major barriers to the \Aedes elimination.
The larval period, which is the feeding and growth stage, depends on temperature, larval density, and availability of food, and in optimal conditions, does not exceed five days.
% The larva is divided into thorax, head, and abdomen.
When in low temperature and lack of food, this phase can extend for weeks. %, until they become pupae.
The pupa is a phase without nutriment
% and, besides, the transformation to the adult stage~\cite{lambrechts2012vector}.
% To pass
from the embryo to the adult stage (egg, larva, and pupa), which takes on average ten days.
On the first or second day after becoming adults, mosquitoes copulate.
After mating, females begin to feed on blood since it has the necessary proteins for the development of eggs~\cite{lambrechts2012vector}.


% As can be observed, only the female mosquitoes feed on blood, in order to the maturation of the eggs.
% Thus, t
The arbovirus can be maintained in the \Aedes populations by transovarial transmission in which the female vector passes the infectious agent through the eggs to the next generation~\cite{monath1994dengue}.
Such transmission is epidemiologically important, as epidemics of dengue, for example, usually enable quick proliferation.
That depends on the time it takes for the mosquito to become vector after it stings a viremic person.
If mosquitoes already emerge as vectors, \ie infected and capable of transmitting the virus, without the need to prick an infected person, the probability of vector-human-vector transmission increases, as well as the magnitude of the epidemic.
The transovarial transmission also creates the possibility that the \Aedes eggs, which carry the virus, spread to other geographic regions by being passively transported~\cite{monath1994dengue}.

Like other species, \Aedes is also particularly sensitive to climatic conditions.
Several studies demonstrate the role that factors such as temperature and precipitation have on ecology and vector biology~\cite{morin2013climate}.
% The temperature interferes directly in several aspects related to the biology of the vector~\cite{morin2013climate}.
It is verified that the higher the temperatures, the faster the development of the different phases of the mosquito, and the higher its longevity and fecundity during the adult phase; at lower temperatures, there is a degraded condition to its development, which may jeopardize survival.
Concerning precipitation, it favors the creation of potential breeding sites where females lay their eggs and immature forms (larvae and pupa) develop~\cite{Halstead2007dengue}.

According to~\cite{focks1995simulation}, the vector population grows with increasing temperatures. At 32°C the number of mosquito bites is twice as high as at 24°C.
Another interesting aspect related to temperature is that the rate of metabolism of the mosquito, its evolutionary cycle, can be extended up to about 22 days in the cold months.
On the other hand, in the months of high temperatures, its maturation speed increases.
Therefore, the ambient temperature is directly proportional to the time of development of the vector for the adult phase.
Nonetheless, temperatures above 40°C decreases the life expectancy of the mosquito~\cite{patz1998dengue}.
Researches diverge about the relationship between rainfall and the increase of the presence of vectors in the environment.
Excessive rainfall could lead to the decline of breeding sites due to flooding, whereas lack of rainfall may lead to the storage of water in domestic reservoirs (vases, dishes, canisters, etc.) and in other locals that can be used for vector reproduction (barrels, wells, water boxes and etc.).
However, in abandoned areas with poor structural and sanitary development, rainfall incidence is important to understand the development of breeding sites in fixed deposits (gutters, slabs, glass shards in walls, and other architectural works), solid waste, and other abandoned items.
For that reason, there are different correlations between precipitation and the \Aedes proliferation along the geographical spaces, being possible to occur positive or negative relations~\cite{arcari2007regional}.

Finally, the understanding of these biological and ecological aspects of the vector is of great epidemiological importance, since studies of this nature generate information about the reproductive, hematophagy dynamics and vectorial competence, and help to understand the mechanism of arbovirus transmission.
This knowledge can assist in the creation of mechanisms for combating and controlling the \Aedes.



\section{Diseases transmitted and sequelae}
%
The \Aedes is a transmitter of some diseases, known as arboviruses.
Nevertheless, it is important to note that only infected mosquitoes transmit the disease.
The main diseases are dengue, zika, chikungunya, and yellow fever.
This section presents, in a concise way, a panorama of these diseases, their transmission, and their possible sequels.

\subsection{Dengue}

Dengue is the most rapidly spreading mosquito-borne viral disease in the world.
During the last 50 years, the incidence of dengue has increased 30 times, and the number of affected countries has been increasing steadily~\cite{rigau1998dengue}.
Today, approximately 3.6 billion people live in more than 100 dengue-endemic countries, and an estimated 284-528 million dengue infections occur annually~\cite{guzman2010dengue}.
It is estimated that 500,000 people with severe dengue require hospitalization each year and about 2.5\% of those affected with, die~\cite{web:who2014dengue}.

Dengue fever is an acute febrile disease caused by the arbovirus (arthropod-borne virosis) that has the \Aedes as the vector transmission.
Dengue virus has four serotypes (DENV-1, DENV-2, DENV-3, and DENV-4) that are genetically and antigenically distinguishable.
In Brazil, the four types of virus circulate, being the last one isolated in the State of Roraima since 1991~\cite{teixeira2009dengue}.

This disease can last up to seven days, evolving into spontaneous healing or leaving sequelae.
It can manifest itself in dangerous ways, as is the case of hemorrhagic dengue.
In a case of suspected classical dengue fever, the patient is diagnosed with acute febrile illness lasting up to seven days accompanied by at least two of the following symptoms: headache; pain around the eyes, in the body and joint; and prostration.
%
The hemorrhagic dengue usually arises, most of the time, when the person is infected more than once by the virus, leading to changes in blood clotting.
Therefore, it causes bleeding especially in the eyes, gums, ears, and nose, as well as the appearance of blood in the stool, red skin patinas, vomiting, weak and fast pulse.
Classical dengue fever is confirmed by laboratory tests; however, during epidemics, confirmation can be performed through clinical-epidemiological criteria.
On the other hand, cases of hemorrhagic dengue need to be confirmed by the laboratory, as well as by specific criteria~\cite{brasil2002dengue}.

Dengue vectors become infected with the virus by feeding on the blood of individuals in the viremia stage.
This phase begins one day before the fever and lasts six to eight days after the onset of the disease.
In the mosquito, the virus multiplies in its cycle of evolution, discussed in Section~\ref{sec:bio}. This is the extrinsic period, after maturation of the mosquito in which it becomes a vector, transmitting the virus through the saliva throughout its lifetime~\cite{watts1987effect}.

\subsection{Zika}
%
Zika is also an arbovirus transmitted by {\it Aedes aegypti}, being first identified in Brazil, April 2015.
The zika virus was given the same name as the place of origin: the Zika forest, in Uganda; identified in sentinel monkeys used to monitor yellow fever, in 1947.
The zika virus disease presents a higher risk than other arboviruses, such as dengue, yellow fever, and chikungunya, due to the development of neurological complications such as encephalitis and Guillain Barré syndrome~\cite{petersen2016zika}.
One well-known neurological complication is the microcephaly; a condition in which the baby's head is smaller than the average.
% head of children of the same age and same sex.
It usually happens when there are problems in the uterus that causes the baby's brain to stop growing properly, which may also occur after birth.
Microcephaly-born children often present developmental difficulties.
Rarely, children with this condition can develop normally. In addition to congenital microcephaly, many manifestations have been reported among infants up to four months of age exposed to the zika virus in the uterus.
These include head malformations, involuntary movements, seizures, irritability, and brain stem dysfunction, with swallowing problems, limb contractures, hearing and vision abnormalities, and brain abnormalities.
Other consequences associated with zika virus infection in the uterus may involve spontaneous abortions and stillbirths. The spectrum of congenital abnormalities associated with fetal exposure to this virus during gestation is known as congenital zika virus syndrome~\cite{web:who2016zika}.

The symptoms of zika are red spots all over the body, red eye, fever, body aches and joints of small intensity.
In general, the disease progresses benign, and the symptoms disappear spontaneously after 3 to 7 days.
However, joint pain may persist for about a month.
According to the Ministry of Health, all sexes and age groups are susceptible to the zika virus; however, pregnant women and older adults are at higher risk of developing the complications aforementioned.
These risks are amplified when the person has some chronic illness, such as hypertension and diabetes, even if treated~\cite{PAHO2017zika}.

Thus, like dengue, the primary mode of transmission is by the bite of the vector, but it can also be transmitted through sexual intercourse.
According to the Pan American Health Organization (PAHO),
\abbrev{\id{PAHO}PAHO}{Pan American Health Organization}
the zika virus can be found in semen, blood, urine, amniotic fluid, and saliva as well as fluids found in the brain and spinal cord~\cite{PAHO2017zika}.

The zika's diagnosis is based on the patient's new symptoms and history (such as mosquito bites or trips to areas with virus circulation).
Moreover, laboratory tests may confirm the presence of zika in the blood; however, this diagnosis may not be as reliable as the virus could react with other viruses such as dengue and yellow fever~\cite{PAHO2017zika}.

\subsection{Chikungunya}
%staples2009chikungunya
Chikungunya fever is a viral disease transmitted by the mosquitoes \Aedes and \textit{Aedes albopictus}, and its circulation was first identified in Brazil in 2014.
Chikungunya means ``those who fold'' in Swahili, one of the languages of Tanzania, located in East Africa.
That refers to the curved appearance of patients who were seen in the first documented epidemic there, between 1952 and 1953~\cite{staples2009chikungunya}.
The main symptoms are rapid onset fever, intense pain in the joints of the feet and hands, in addition to fingers, ankles, and wrists.
There may also be a headache, muscle aches and red spots on the skin.
It is not possible to have chikungunya more than once.
Previously infected, a person becomes immune to lifetime.
The symptoms begin between two and twelve days after the mosquito bite.
The mosquito gets the CHIKV virus by stinging an infected person during the period the virus is present in the infected organism.
The incubation period of the virus in the human is 4 to 7 days~\cite{staples2009chikungunya}.
The chikungunya can also be transmitted from the pregnant woman to the fetus, but this only occurs when the mother becomes ill in the last week of gestation.
In this case, the child who is born healthy remains hospitalized for observation and immediate treatment. This procedure is adopted because if the disease develops, the child may present severe pictures with neurological and skin manifestations.
Besides, some factors contribute to the enduring complications of the disease, as advanced age, being a woman and already possess other diseases such as diabetes and rheumatoid arthritis.
The CHIKV virus can generate lasting sequelae such as persistent inflammation in the joints, especially the hands and feet.
Although joint pain is the most frequent chronic complication, it is not the only one.
There is a possibility that the virus can trigger neurological problems such as Guillain Barré syndrome, encephalitis, and other complications~\cite{web:who2017chik}.

\subsection{Yellow fever}
%
Yellow fever is an acute febrile infectious disease, caused by a virus transmitted by mosquito vectors, and has two cycles of transmission: wild (when there is transmission in rural or forest areas ) and urban.
There is no direct transmission from person to person.
Yellow fever is epidemiologically important because of its clinical severity and potential for dissemination in urban areas infested by the \Aedes.
Early symptoms of yellow fever include sudden onset of fever, chills, severe headache, back pain, general body aches, nausea and vomiting, fatigue and weakness. Most people get better after these initial symptoms. However, according to the Ministry of Health, about 15\% have a brief period of hours a day without symptoms and then develop a more severe form of the disease. In severe cases, the person may develop a high fever, jaundice (yellowing of the skin and whites of the eyes), bleeding (especially from the gastrointestinal tract), and eventually multiple organ failure and shock. About 20\% to 50\% of people who develop severe illness might die~\cite{web:MSyellowfever}.

The vaccine is the primary tool for prevention and control, unlike the diseases previously analyzed; the Brazilian government offers the vaccine against yellow fever for the population through the Sistema Único de Saúde (SUS) -- the Brazilian public healthcare.
As aforementioned, there are two different epidemiological cycles of transmission, the wild and the urban.
The disease has the same characteristics from the etiological, clinical, immunological and pathophysiological point of view. In the wild cycle of yellow fever, nonhuman primates (monkeys) are the main hosts and amplifiers of the virus. The vectors are mosquitoes with strictly wild habits, with the genera Haemagogus and Sabethes being the most important in Latin America. In this cycle, man participates as an accidental host when entering forest areas. In the urban cycle, man is the only host with epidemiological importance and transmission occurs from infected urban vectors, the \Aedes\cite{camara2011dynamic}.

\section{Potential grounds, reproduction, and indicators}
%As previously discussed, the \Aedes lives close to the humans.
As could be observed, both the persistence and the progression of the arboviruses are conditioned to the survival and reproduction of their vector in the environment.
Therefore, in the context where there is no vaccine for the diseases presented, with exception of yellow fever, the best prevention is to avoid the vector proliferation.
The presence of this mosquito specie is more common in urban areas, and the infestation is more intense in regions with a high population density and low vegetation, where females have more opportunities for food and have more places to spawn~\cite{lambrechts2012vector}.
Another critical factor is the lack of infrastructure of some localities.
Without a regular supply of water, residents need to store it in large containers that do not receive the necessary care and end up becoming mosquito outbreaks, because they are not entirely closed.
Therefore, efforts to control mosquito proliferation are indeed related to government measures and population commitment.

Since 2017, the Brazilian Ministry of Health has issued Resolution No. 12, which makes it obligatory for entomological surveys of the \Aedes infestation by municipalities and the sending of information. Monitoring the numbers and geographical distribution of mosquitoes over time helps in making timely decisions about how best to manage vector populations.
Surveillance can be used to identify areas with a mosquito-borne high-density infestation or periods in which its population has increased~\cite{brasil2017msresolution}.
Thus, the \Aedes Rapid Index Survey (LIRAa, from Portuguese \textit{Levantamento Rápido de Índices para \Aedes})~\cite{brasil2013LIRAa} is a methodology used by the Brazilian Ministry of Health that allows analyzing, sampling by sampling, the quantity of real estate with the presence of containers with mosquito larvae. The results obtained by this methodology allow the managers to evaluate vector control activities, besides indicating the most used deposits by \Aedes.
The LIRAa classifies the deposits considered as potential breeding sites for the \Aedes in five groups, in order to inform their epidemiological importance, facilitating the targeting of vector control and surveillance actions, as shown in Table~\ref{tab:deposits}.
%
% Group A (water storage), which are grouped in: A1 (high water tank connected to the public water supply and / or to the mechanical collection system in a well, cistern or water mine - water tanks, drums and masonry); and A2 (land-based storage tanks - barrel, drum, barrel, tub, clay tanks, cisterns, water tanks and well water collection). Group B are mobile containers, such as vases/ jars with water, dishes, returnable bottles, drippings, frost-proof containers in refrigerators, drinking fountains in general, small ornamental fountains, building deposit materials (stocked toilets, pipes, etc.) and religious objects. Group C includes fixed deposits, such as tanks in construction sites, drills and kitchen gardens, rails, uneven slabs, and awnings, drains, disused toilets, untreated pools, ornamental fountains; shards of glass in walls, other works, and architectural adornments. Group D consists of those removable: D1 (tires and other materials such as inner tubes); and D2 are a solid waste (plastic containers, pet bottles, cans), scrap and construction debris.
% Finally, group E belong to natural breeding sites such as leaf armpits (bromeliads, etc.), tree and rock holes, and animal remains1
%
\begin{table}[b!]
\caption{The classification of potential breeding sites for the {\it Aedes aegypti}, according to LIRAa.}
\label{tab:deposits}
\begin{tabular}{cl}
\toprule
\textbf{Code} & \multicolumn{1}{c}{\textbf{Description}}\\ \hline
A1 & Water tank connected to the grid (high tanks)\\
A2 & Deposits at ground level (barrel, tub, drum, tank, well) \\
B  & Mobile containers (vases/jars, plates, drippings, drinking fountains, etc) \\
C  & Fixed deposits (tanks, gutters, slabs, etc) \\
D1 & Tires and other rolling materials \\
D2 & Garbage (plastic containers, bottles, cans, scraps) \\
E  & Natural deposits (bromeliads, bark, tree holes) \\ \bottomrule
\end{tabular}
\end{table}



The state of Rio de Janeiro epidemiological report of 2018 shows that type A2, B, C, and D2 deposits account for 84.7\% of the 5,608 breeding sites found in the whole territory.
According to the National Guidelines for Prevention and Control of Dengue Epidemics (2009), the parameters for classification of municipalities regarding \Aedes infestation are (i) satisfactory if less than 1\%; (ii) alert if between 1\% and 3.99\%; and (iii) risk if above 3.99\%.
From all 92 municipalities in Rio de Janeiro, 91 (98.9\%) performed the LiRAa.
From these, 45 (49.5\%) are classified as satisfactory, 43 (47.3\%) in the alert, and 3 (3.3\%) in risk~\cite{rj2018epid52018}.

% At national level, the LIRAa indicate that 504 municipalities have a high rate of infestation, with a risk of epidemics of diseases transmitted by \Aedes.
% Moreover, the LIRAa identified 1,881 municipalities on alert, with a building infestation index (IIP, from portuguese \textit{Índice de Infestação Predial}) between 1\% and 3.9\%, and 2,628 municipalities with satisfactory indexes.
% In addition, the reports of 2018 show that in the Southeast the largest number of deposits found was in households, characterized by vessels/bottles with water, dishes, and returnable bottles.
% In the Midwest, North, and South, garbage predominated, such as plastic containers, pet bottles, cans, scrap, and construction debris~\cite{web:brasil2018liraa}.

At the national level, 5.358 municipalities, 96.2\% of the total, performed some transmitter monitoring.
The Brazilian Ministry of Health indicates a reduction in the three diseases transmitted by the \Aedes between January and October 2018, compared to the same period in 2017; however, some states show a significant increase in cases of dengue, zika, and chikungunya.
In 2018, 241,664 cases of dengue were reported over the country, there was a small increase compared to the previous year (232,372).
Fortunately, the number of deaths reduced from 176 to 142.
%
Besides, there was a reduction of 54\% concerning the previous year in the notification of cases of chikungunya, from 184,344 to 84,294 in 2018.
The number of deaths followed this drop, reducing from 191 to 35, a significant fall of 81.6\%.
%
Concerning to zika, there was also a reduction in the number of cases, from 17,025 to 8,024~\cite{web:brasil2018liraa} and four deaths were reported.

The difficulty of mosquito control in Brazil is the non-uniformity of compliance with the guidelines of the dengue control program, zika and chikungunya in all municipalities, as well as the inability of epidemiological and entomological surveillance to eliminate all possible existing outbreaks (breeding sites) in all regions of all Brazilian cities.

\section{Plans and actions to combat the vector}
%
Dengue, zika, and chikungunya are vector-borne diseases, and due to the lack of vaccines and specific antiviral drugs, up to now, the prevention of these diseases are the control, elimination or eradication of the vector.
%
% The control of the \Aedes has been defined as the integrated and selective use of the different methods of vector combat in an efficient, economical and safe way, in order to reduce the population and the transmission to acceptable levels.
% The WHO have not defined these levels quantitatively~\cite{aaa?}.
%
The interventions aimed at vectorial control have been based on two sets of actions: chemical control and environmental management.
The chemical combat allows acting in the larval or adult forms of the vector.
Larvae are eliminated in their habitat (accumulated water) by the use of larvicides, while winged forms are eliminated by spraying the environment with pyrethroid or organophosphorus insecticides at “ultra-low volume”~\cite{araujo2015aedes}.
Promising attempts have been made to use biological methods of larval control.
Environmental management in current national control programs has generally been limited to the destruction of potential breeding sites in the domiciles and peridomiciles environments, without intervening in other elements of the urban infrastructure (garbage collection, water supply, etc.) and the way of the population lives~\cite{araujo2015aedes}.

An alarming fact is that in 2016, 3.31 million cases of dengue were registered on the planet and reported to the WHO, almost half of them in Brazil.
This number was the highest in recent history.
Taking into account that it does not include data from Africa, where the infrastructure to monitor the incidence of the disease is more precarious, the global number would be even higher.
Therefore, the data show the potential need for a multi-scalar \Aedes control strategy, both locally and globally.
In order to combat the mosquito, the WHO presented a comprehensive strategy of vectorial control in 2004~\cite{world2004global}; it is a way of integrated management, in which Brazil is a member.
% While this approach seeks to improve the efficacy, cost-effectiveness, ecological soundness, and sustainability of disease-vector control, uptake has been red, due to insufficient political buy-in for reorientation of programs to support a harmonized approach to vector control across diseases~\cite{araujo2015aedes}.
% This has been mainly due to fragmented global and national architecture to support a multiscalar approach.
%
The WHO strategy builds on the basic concept of integrated vector management with a renewed focus on improving governance capacity at the national and subnational levels.
There is an emphasis on strengthening infrastructures and systems (\eg sustainable development, access to potable water, adequate solid waste and excreta management), particularly for vulnerable areas~\cite{world2012global}.
As pointed by the WHO, for a sustainable impact on vector control, greater intersectoral and interdisciplinary action is needed, linking efforts in environmental management, health education, and reorienting relevant government programs around proactive strategies to control new and emerging threats~\cite{world2012global}.
Therefore, critical attention is adapting to local contexts these answers and global strategies to make effective and sustainable control of the vector.

Although several countries in the Americas have been considered “free” of the \Aedes in the recent past, the discussion about the difficulties or even the impossibility of eradicating a biological species from a large geographical area remains alive.
In 2001, the Brazilian government began to consider vector control instead of the eradication goal, with the implementation of the Dengue Control Actions Intensification Plan (PIACD), prioritizing actions in municipalities with the higher transmission of dengue. In 2002, the Brazilian National Plan for Dengue Control (PNCD) was developed due to the increased risk of epidemics, the occurrence of severe dengue cases and the reintroduction and rapid dissemination of serotype 3 in the country~\cite{web:pncdbrasil}.
The PNCD also seeks to incorporate lessons from national and international dengue control experiences, emphasizing the need for change in previous models, \ie it is a reflection of the new global approach incorporated in WHO.

Furthermore, with the support of the Brazilian Ministry of Health and the states, the municipal health secretariats began to manage and implement PNCD actions.
These actions involved ten main components: epidemiological surveillance, vector control, patient care, integration with basic care, environmental sanitation actions, integrated actions of health education, communication and social mobilization, training of human resources, legislation, political and social support, and monitoring and evaluation of PNCD~\cite{web:pncdbrasil}.
Thus, the program was no longer exclusively directed at combating the vector and suggested adaptations consistent with local specificities, including the possibility of elaborating sub-regional plans.

In Brazil, the community, health and endemics combat agents, in partnership with the population, are responsible for promoting the control of the vector, whose actions are focused on detecting, destroying or properly allocating natural or artificial water reservoirs that may serve as a deposit for mosquitoes' eggs.
Another complementary strategy advocated by the Brazilian Ministry of Health is the promotion of educational actions during the home visit by the community agents, with the objective of guaranteeing the sustainability of the elimination of the breeding sites, in an attempt to break the chain of transmission of the diseases~\cite{brasil2009diretrizes}.

It is possible to use three types of mechanisms in combating the {\it Aedes aegypti}: mechanical, biological and chemical.
The mechanical control consists in the adoption of practices capable of eliminating the vector and the breeding sites or reduce the contact of the mosquito with humans.
Besides, biological control is based on the use of predators or pathogens with the potential to reduce the vector population.
Finally, chemical control, as the name suggests, consists of the use of chemicals
% , which may be neurotoxic, juvenile hormone analogs and chitin synthesis inhibitors
to kill adult larvae and insects. It is a type of control which the use must be done safely and rationally, complementing the actions of surveillance and environmental management.
The irresponsible use may cause selecting vectors resistant to the products and environmental impacts~\cite{world1996chemicals}.

Thereby, new technologies have been developed as alternatives to the control of the mosquito, using different mechanisms of actions, such as social measures, careful monitoring of infestation, dispersion of insecticides, new chemical and biological control agents and genetic procedures for control of the mosquito populations, including combinations of techniques.
The adoption of vector control strategies combinations requires continuous evaluation of effectiveness~\cite{araujo2015aedes}, considering the possible synergistic effects between compatible strategies and spatial heterogeneity, based on an assessment of risk areas.
Therefore, mapping techniques are also presented as a promising strategy, developed to evaluate and identify areas of increased risk for arbovirus transmission in certain territories using local spatial statistics.
By linking spatial data with entomological surveillance data (characteristics, presence, infestation rates, and efficacy evaluation of control methods), epidemiological surveillance, laboratory network, and sanitation, specific vector control actions are directed to priority areas.

\section{Conclusions}
%
The \Aedes is one of the primary biological vector responsible for transmitting a wide range of arboviruses.
These infections are on the rise and extending to new geographical areas.
As previously discussed, these viruses are causing a tremendous negative impact on the health of the Brazilian and global population.
Thus, controlling the vector mosquitoes is critical for preventing these diseases.
The control methods should be considered according to the regional context including chemical, biological and environmental means.
Furthermore, the governmental effort should be intensified to come up with and stimulate more innovative ways of controlling and surveillance of the mosquito.
In this work we propose a methodology to help in the combat of the {\it Aedes aegypti} by locating locals with potential mosquitoes breeding sites.






