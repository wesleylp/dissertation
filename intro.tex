\chapter{Introduction}
\label{chap:intro}

The \emph{Aedes aegypti} is the main vector of several diseases, such as dengue, zika, chikungunya and, more recently in Brazil, urban yellow fever. 
Zika virus disease can be quite dangerous for pregnant women because of its correlation with the microcephaly, a congenital fetus brain malformation.
Dengue is the one that causes the most deaths, with about 390 million infections per year in the world~\cite{bhatt2013global}.
Yellow fever also has a high rate of lethality and chikungunya can incapacitate those infected for long periods.

By considering the high rates of lethality and eradication difficulty, arboviruses transmitted by \emph{Aedes aegypti} are one of the leading global health problems.
Wherefore, the World Health Organization (WHO) launched, in 2012, a comprehensive strategy for dengue control and prevention~\cite{world2012global}, whose one of the goals is to reduce disease cases by 25\% by 2020. 
Unfortunately, combat tools are still limited: the dengue vaccine remains in the improvement phase, and the fumes against mosquitoes are ineffective~\cite{newton1992model}. 
Thus, the current best form of combat is through the control and elimination of possible mosquito foci proliferation, which acts directly in the prevention of all these diseases. 
Given that the \emph{Aedes aegypti} reproduces in clean and stagnant water, the main mosquito foci are open water bowls, gutters, tires, bottles, plant pots, and any container that can collect water. 


As a result, monitoring and controlling the mosquito without proper technical support is expensive, time-consuming and therefore inefficient. 
For that reason, allying the knowledge of an expert with a tool that accelerates and towards a more precise work is extremely important in the current scenario. 
Thus, using images and videos captured by an unmanned aerial vehicle (UAV), better known as a drone, with several sensors and camera is a reasonable approach.
The objective is to identify objects with high potential of being a mosquito breeding site.
This technology has already been used by organizations to inspect difficult-to-reach sites in order to locate such breeding spots.
In this process, the acquired videos are examined by a specialist, which makes the procedure time-consuming and tiring, what may lead to failures.

In this sense, a possible solution to increase efficiency is to apply machine learning techniques to automate the analysis process, helping the specialist in the decision-making action~\cite{casfinal2018}. 
After this fast analysis, potential breeding sites can be treated or removed by a team of agents, as usual. 
It is known, through a local study~\cite{tun2009reducing}, that identifying the place of highest infestation, and only treating it is almost as effective as treating all the existing ones, drastically reducing the potential for epidemic development. 
In the case of Nova Iguaçu, a city located in the state of Rio de Janeiro, the reservoirs listed with high potential were, according to~\cite{Lagrotta2006}: water tanks, glass and plastic bottles, buckets, tires, and external drains. 
The goal then becomes to automatically recognize as many of these objects as possible in a video or image to provide an intelligent decision support tool for agents, thereby increasing the effective area of action.


\section{Dissertation organization}
%
In Chapter~\ref{chap:mosquito}, we do a review on the {\it Aedes aegypti}, including biologic aspects, transmitted diseases and sequelae, and preferred ground sites.
Also, we point out some statistics related to the theme and government plans to combat the transmitter.
In Chapter~\ref{chap:database}, we make a literature review of related themes, in order to see how several techniques of machine learning can be used to address the problem.
We also present a new dataset containing the main objects considered as potential foci of the \Aedes in several scenarios.
In Chapter~\ref{chap:system}, since we are interested in detecting objects, we describe the state-of-the-art algorithm employed in order to accomplish this task.
We discuss from classical object detections up to recent deep-learning-based models, particularly the one employed in this work.
The evaluation method, implementation details, and results for the method used are discussed in Chapter~\ref{chap:results}; and we finally conclude at Chapter~\ref{chap:conclusions}.


