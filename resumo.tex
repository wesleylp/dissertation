
\hyphenation{chikungunya}
\begin{abstract}
\selectlanguage{brazil}
Todos os anos, milhares de pessoas são afetadas por doenças como dengue, chikungunya, zika e febre amarela.
Todas essas doenças são transmitidas pelo {\it Aedes aegypti}, que se reproduz em água limpa e parada, usualmente acumulada em recipientes como pneus, garrafas, caixas d'água etc.
O uso de ferramentas inteligentes pode auxiliar no trabalho dos agentes de fiscalização dos focos deste mosquito, aumentando, assim, a eficiência e área de cobertura.
Esse trabalho aborda o problema de detecção automática de focos de mosquitos através do uso de técnicas de visão computacional e aprendizado de máquina.
Nesse contexto, propõe-se um conjunto de vídeos aéreos, adquiridos através de um veículo aéreo não tripulado.
O conjunto possui diversos desses objetos em múltiplos cenários:
diferentes localidades, altitudes e disposições dos objetos.
Os vídeos são devidamente retificados para amenizar distorções da câmera e manualmente anotados quadro-a-quadro, viabilizando o desenvolvimento de um detector automático de objetos de interesse.

Um detector do tipo \textit{Faster Region-based Convolutional Neural Network} é treinado com uma pequena base de dados, e é capaz de encontrar possíveis focos de mosquito de maneira automática.
O modelo gerado atinge uma precisão média de 49,31\%, o que é promissor, indicando que novos e melhores modelos podem ser treinados para este fim.

\end{abstract}

